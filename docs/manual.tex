\documentclass{article}
\usepackage[a4paper,bottom=3cm,top=3cm,left=2.5cm,right=2.5cm]{geometry} 
\usepackage{graphicx}
\usepackage[colorlinks=true]{hyperref}
\usepackage{palatino}

% paragraph
\setlength\parindent{0pt} 
\parskip2ex

\begin{document}

\title{RNAcode: Getting started}
\maketitle

\section{Installing RNAcode on Unix like systems (Linux/OS X)}

\texttt{RNAcode} is distributed as \texttt{tar.gz} compressed
archive. Download the package to your machine and uncompress it, e.g.

\begin{verbatim}
# tar -xzf RNAcode-0.3.tar.gz
\end{verbatim}

\texttt{RNAcode} uses the standard GNU installation system. So you can easily
install it by running the following commands:

\begin{verbatim}
# ./configure
# make
# make install (requires root privileges)
\end{verbatim}

This will install \texttt{RNAcode} in the \texttt{/usr/local/} tree. To install
somewhere else run configure like that:

\begin{verbatim}
# ./configure --prefix=/home/stefan/programs
\end{verbatim}

To test if \texttt{RNAcode} was successfully installed, run 

\begin{verbatim}
# RNAcode -V
\end{verbatim}

\section{Obtaining multiple sequence alignments}

To identify a coding region in your sequence of interest, you need one
or more homologous sequences and create a multiple sequence
alignment. The more sequences in your alignment the better the
accuracy. However, if sequences are too similar ($>90$\%) identity
they only contribute little new information. Our benchmarks showed
that alginments with 5--10 sequences and amean pairwise identity
$<90$\% give good results.

There are different possible scenarios how to obtain alignments for
your sequences.

\subsection{Download pre-made alignments}

These days, for many organisms the complete genomic sequence is
known. Moreover, for many organisms also related species have been
sequenced and pre-calculated multiple alignments are available for
download. Well known resources for major model organism are for
example \url{genome.ucsc.edu} or \url{ensemble.org}, but also many
independent smaller genome projects provide multiple alignments.

\subsection{Create alignments of long genomic regions}

If you want to analyze longer genomic regions ($>1kb$) but cannot find
pre-made alignments, we recommend using the \texttt{MultiZ} program
suite to create alignments. It can be downloaded here
\url{http://www.bx.psu.edu/miller_lab/dist} and comes with excellent
documentation. 

You will need homologous genomic sequences for your region in other
species. For example, you can align the complete genomes of bacteria
or align homologous loci in the megabase range of higher organisms
with \texttt{MultiZ}.

\subsection{Create alignments of individual short regions}

If your sequence of interest is relatively short (a few hundred
nucleotides) we recommend using a simple global alignment program like
for example \texttt{ClustalW}. Use \texttt{Blast} to find homologous
sequences in a sequence database (e.g. GenBank,
\url{www.ncbi.nlm.nih.gov/genbank/}). Collect the significant hits
that match to your region of interest and align the sequences
afterwards with an alignment program.

\section{Formatting the alignments}

\texttt{RNAcode} can process alignments in two different formats: MAF
and CLUSTAL W. You have to make sure that your alignment is in one of
these formats before you can use \texttt{RNAcode}.

The MAF format was popularized by the USCS genome browser and is very
useful to represent genome-wide alignments. The detailed specification
can be found here: \url{http://genome.ucsc.edu/FAQ/FAQformat.html}. If
you download alignments from an UCSC resource it is usually formatted
as MAF. Also if you align your sequences using \texttt{MultiZ} the
default output format is MAF.

For shorter alignments of individual regions, the CLUSTAL W format is
useful. Apart from CLUSTAL W, many other alignment programs output
their alignments in this format.

\section{Pre-processing alignments}

If your alignments contain blocks of long genomic regions it is
usually no reasonable to score these long regions as a whole. The
\texttt{tar.gz} package contains a script \texttt{breakMAF.pl} that
allows you to easily pre-process your MAF files:

\begin{verbatim}
# scripts/breakMAF.pl examples/genomic.maf > genomic-preprocessed.maf
\end{verbatim}

This command breaks blocks longer than 400 in shorter blocks of an
average size of 200.

\section{Running RNAcode}

Analyze alignment with standard options and print detailed results
page:

\begin{verbatim}

# RNAcode examples/coding.aln

\end{verbatim}

Analyze alignment and show best non-overlapping hits below a $p$-value
cutoff of 0.01 in gtf format:

\begin{verbatim}
# RNAcode --outfile out.gtf --gtf --best-only --cutoff 0.01 genomic-preprocessed.maf
\end{verbatim}

Create color annotations for high scoring coding segments:

\begin{verbatim}

# RNAcode --eps coding.aln

\end{verbatim}

Please refer to the more detailed README file that explains all
options of \texttt{RNAcode} and how to interpret the results.

For details on the methodology refer to the paper:

\textbf{RNAcode: robust discrimination of coding and noncoding regions in
comparative sequence data}\\
Washietl S, Findei\ss\ S, M{\"u}ller S, Kalkhof S, von~Bergen M, Hofacker IL, Stadler PF, Goldman N\\
\emph{RNA} (2011), in revision

\end{document}